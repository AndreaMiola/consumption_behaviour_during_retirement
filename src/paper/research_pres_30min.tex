\documentclass[11pt]{beamer}
% \documentclass[11pt,handout]{beamer}
\usepackage[T1]{fontenc}
\usepackage[utf8]{inputenc}
\usepackage{float, afterpage, rotating, graphicx}
\usepackage{epstopdf}
\usepackage{longtable, booktabs, tabularx}
\usepackage{fancyvrb, moreverb, relsize}
\usepackage{eurosym, calc}
\usepackage{amsmath, amssymb, amsfonts, amsthm, bm}

\usepackage{natbib}
\bibliographystyle{rusnat}


\hypersetup{colorlinks=true, linkcolor=black, anchorcolor=black, citecolor=black, filecolor=black, menucolor=black, runcolor=black, urlcolor=black}

\setbeamertemplate{footline}[frame number]
\setbeamertemplate{navigation symbols}{}
\setbeamertemplate{frametitle}{\centering\vspace{1ex}\insertframetitle\par}


\begin{document}

\title{Consumption Behaviour during Retirement}

\author[Andrea Miola]
{
{\bf Andrea Miola}\\
{\small University of Bonn}\\[1ex]
}


\begin{frame}
    \titlepage
    \note{~}
\end{frame}


\begin{frame}[t]
    \frametitle{Introduction}
    \begin{itemize}
        \item Setting the saving and knowing the consuption amount for the retirement is really difficult
        \item Neoclassical theory: consumption should be smoothed across periods (Modigliani, 1954).
        \item Falling in consumption as "consumption puzzle" (Banks et al., 1998; Bernheim et al., 2001; Schwerdt, 2005).
        \item Falling as a result of changes in the opportunity cost of time after retirement (Aguiar and Hurst 2005, 2013).
        \item The aim of this work is to replicate the results of Haider and Stephen (2007) using SHARE data.
   \end{itemize}
    \note{~}
\end{frame}


% Print black screen only in presentation mode for finishing up.
\begin{frame}[t]
    \frametitle{Data description}
    \begin{itemize}
        \item The Survey of Health, Ageing and Retirement in Europe (SHARE)
        \item The panel cover 7 waves from 2004 to 2017.
        \item The initial sample includes around 43.000 individuals, male and female, born between 1923 - 1978.
        \item I selected only data related to employment and pension status, consumption behaviour and demographics characteristics.
        \item Retirement is based on the current labor force status. Dummy: 1 if retired 0 otherwise.
        \item Restritions: on working on the initial wave, expected retired indicate.
   \end{itemize}
    \note{~}
\end{frame}

\begin{frame}[t]
    \frametitle{Estimation strategy}
    \begin{itemize}
        \item I used Least square dummy variable estimation (LSDV) for panel data
        \item Regression equation as:
            \begin{equation}
                \label{1stequation}
                \Delta ln C_{t+1} = \alpha + \beta retire_{t+1} + \gamma X_{t+1} + \nu_{t+1}
            \end{equation}
        \item I used only food consumption in home household.
        \item Conditional independence assumption holds conditionally on the controls.
   \end{itemize}
    \note{~}
\end{frame}

\begin{frame}[t]
    \frametitle{Regression results}
    \begin{itemize}
        \item The results are negative but non significant
        \item However, the coefficients show a drop in consumption at the retirement period
        \item Covariates are consistent with the controls used in different studies, as I expected.
   \end{itemize}
    \note{~}
\end{frame}

\begin{frame}[t]
    \frametitle{Heterogeneity analysis}
    \begin{itemize}
        \item The figure 1  below shows how the parameter of interest varis across the group for the observed retired variable.
            \begin{figure}[h]
                \centering
                \includegraphics[width=0.70\textwidth]{../../out/figures/CI_job_Retired.pdf}
                \caption{Confidence Interval of Observed Ret. per years classes.}
            \end{figure}
   \end{itemize}
    \note{~}
\end{frame}

\begin{frame}[t]
    \frametitle{Heterogeneity analysis}
    \begin{itemize}
        \item fugure 2 shows the variation of the parameter of interest for the expected retirement year.
            \begin{figure}[h]
                \centering
                \includegraphics[width=0.70\textwidth]{../../out/figures/CI_exret.pdf}
                \caption{Confidence Interval of Observed Ret. per years classes.}
            \end{figure}
   \end{itemize}
    \note{~}
\end{frame}

\begin{frame}[t]
    \frametitle{Conclusions}
    \begin{itemize}
        \item This consumption decline is a puzzle if it is observed for workers who retire when expected.
        \item The results are not significant.
        \item However, main aim of this project was to create a reproducible work.
        \item Future improvements would be absolutely necessary.
   \end{itemize}
    \note{~}
\end{frame}

\begin{frame}[allowframebreaks]
    \frametitle{References}
    \nocite{*}
    \bibliography{refs}
\end{frame}

\end{document}
